\chapter{Metodología}
\label{cap:metodologia}

En este apartado se va a describir la metodología utilizada durante la realización de este proyecto. A su vez, se va a incluir una descripción de las herramientas, tecnologías y técnicas usadas, asi como de una justificación de dichas elecciones.


\section{Metodología}

Al comienzo de este proyecto se tuvieron varias reuniones para establecer los objetivos y requisitos que debía de cumplir este proyecto, aunque no en demasiada profundidad, siendo esto una tarea que se llevo a cabo según lo establecido en el capítulo anterior.\\

Una vez finalizadas dichas reuniones, se establecieron una serie de fechas en las cuales se tuvieron otras reuniones para comprobar el estado del proyecto. Estas fechas coinciden con la planificiación. De esta manera, se intercambiaron opiniones sobre la situación del proyecto, comentando posibles mejoras y solucionando las dudas surgidas durante las distintas etapas del desarrollo.\\

\section{Herramientas usadas}

En esta sección vamos a comentar las herramientas que han sido usadas durante la realización del proyecto.

\subsection{Android Studio}

Aunque no es estrictamente necesario para desarrollar aplicaciones Android, es posible utilizar eclipse por ejemplo, he decidio usarlo ya que es muy sencillo de entender y facilita al progamador muchas tareas. Esto ha sido muy importante ya que, como he comentado antes, mi experiencia con Android era muy reducida.\\

Android Studio es el entorno de desarrollo integrado, \textit{IDE}, oficial para la plataforma Android. Por esta razón la documentación es abundante, lo que supone un gran punto a su favor.\\

\begin{figure}[H] %con el [H] le obligamos a situar aquí la figura
\centering
\includegraphics[scale=0.5]{imagenes/android-studio.jpg}  %el parámetro scale permite agrandar o achicar la imagen. En el nombre de archivo puede especificar directorios
\label{android-studio.jpg}
\caption{Ejemplo de proyecto Android Studio}
\end{figure}

Otra de las cosas interesantes de este \textit{IDE} es la posibilidad de diseñar interfaces de una manera muy sencilla e intuitiva, permitiendo arrastar los elementos a las posiciones deseadas. Por lo que no se requiere un gran nivel de programación para estas tareas. Aunque si hay que comentar, que si se necesita hacer cosas más complicadas, o que no sean las estándar, si es necesario un nivel de programación avanzado, ya que en dicho caso, la ayuda proporcionada por Android Studio para estas tareas se reduce.\\ 

\subsection{Git y GitHub}

Al tratarse de un proyecto de esta magnitud, ha sido necesario utilizar una herramienta de control de versiones, como es natural se ha utilizado git.\\

También se ha usado GitHub, que se trata de un lugar donde alojar nuestros proyectos utilizando el sistema de control de versiones Git. Por lo tanto, podemos entender que git y GitHub van de la mano, al menos en este caso.\\

El respositorio del proyecto se puede consultar \href{https://github.com/acasadoquijada/jmr-android}{aquí}.

Para llevar un control del proyecto se han usado los elementos conocidos como \textit{Milestones} y \textit{Issues} por GitHub.\\

Podemos entender un \textit{Issue} como una tarea por realizar, siendo un ejemplo, \textit{seleccionar imagen de la galería}. Se les puede añadir información extra, como a que \textit{Milestone} está asociado, que persona es la encargada de solucionarlo, o se puede añadir una etiqueta para establecer el tipo.\\

\begin{figure}[H] %con el [H] le obligamos a situar aquí la figura
\centering
\includegraphics[scale=0.4]{imagenes/issue.png}  %el parámetro scale permite agrandar o achicar la imagen. En el nombre de archivo puede especificar directorios
\label{issue.png}
\caption{Ejemplo de issue}
\end{figure}

Por otro lado, se encuentran los \textit{Milestones}, que podemos considerarlos como hitos, es decir, un \textit{Milestone} está compuesto por varios \textit{Issues}. Por lo que también puede ser vistos como una agrupación de \textit{Issues}, una gran tarea dividida en pequeñas subtareas.\\

\begin{figure}[H] %con el [H] le obligamos a situar aquí la figura
\centering
\includegraphics[scale=0.4]{imagenes/milestone.png}  %el parámetro scale permite agrandar o achicar la imagen. En el nombre de archivo puede especificar directorios
\label{milestone.png}
\caption{Ejemplo de milestone}
\end{figure}

Como se puede entender, usar ambos es de vital importancia si se desea llevar a cabo un proyecto de gran magnitud.
hablar sobre mi movil, carácterisiticas y tal

\subsection{Dispotivio de pruebas}

Aunque Android Studio nos ofrece la posibilidad de utilizar un emulador para lanzar la apliación, he decidido utilizar mi dispositivo movil, por motivos de eficiencia y comodidad. Debido a que no podía comprobar de manera real algunos aspectos de la propia aplicación, como el consumo de memoria o el propio rendimiento de las consultas.\\

Mi smartphone es un \textit{Xiaomi redmi 4 pro}, y cuenta con las siguientes características destacables:

\begin{itemize}
\item CPU: Qualcomm Snapdragon 625, con ocho núcleos y 2GHz 
\item Memoria: 3 GB
\item Almacenamiento: 32 GB
\end{itemize}

Se tratan de unas características que suelen ser habituales de encontrar en los smartphones actuales, por lo que ha sido un gran sujeto de pruebas.

\section{Tecnicas}

Para programar en Android se utiliza \textit{Java} para la lógica, y \textit{XML} para las interfaces de usuario.\\

Como es habitual, el desarrollo en Java ha seguido un paradigma de programación orientada a objetos, \textit{POO}. En el que se han desarrollado una serie de clases que se han organizado en distintos paquetes. Esto será explicado posteriormente.\\ 

Comentar que se ha utilizado también \textit{Android NDK}. Se trata de un conjunto de herramientas que nos permiten implementar partes de la aplicación en código nativo, como \textit{C} o \textit{C++}. Esta opción es perfecta para usar en los cálculos que realiza la aplicación. Como en el caso anterior se comentará con más detalle en su correspondiente sección.\\

\begin{figure}[H] %con el [H] le obligamos a situar aquí la figura
\centering
\includegraphics[scale=0.6]{imagenes/ndk.png}  %el parámetro scale permite agrandar o achicar la imagen. En el nombre de archivo puede especificar directorios
\label{ndk.png}
\caption{Ejemplo de código Android ndk}
\end{figure}

Por último comentar que se ha usado \textit{XML} para las interfaces de usuario, la mayor parte del tiempo apoyándose en el soporte proporcionado por Android Studio, pero que a la hora de realizar cosas mas complejas se ha tenido que escribir manualmente dicho código XML.

\begin{figure}[H] %con el [H] le obligamos a situar aquí la figura
\centering
\includegraphics[scale=0.6]{imagenes/interfaz-android-studio.jpg}  %el parámetro scale permite agrandar o achicar la imagen. En el nombre de archivo puede especificar directorios
\label{interfaz-android-studio.jpg}
\caption{Ejemplo de proyecto Android Studio}
\end{figure}
