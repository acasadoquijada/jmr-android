\chapter{Objetivos}
\label{cap:objetivos}

Como todo proyecto que se realiza, este ha de tener una serie de objetivos que justifiquen su realización. Por lo que en este apartado se van a discutir los objetivos de este, diviendose en objetivos principales y secundarios.\\

Los principales son los objetivos que el proyecto debe cumplir completamente, mientras que los secundarios son objetivos que el proyecto no debe cumplir integramente, pero que en caso de cumplirlos, suponen un gran valor añadido al proyecto. Cabe destacar que en este caso, los objetivos secundarios han sido satisfechos en su totalidad.\\

\section{Objetivos principales}

\begin{itemize}
  \item Desarrollar un sistema de recuperación de imágenes para plataformas móviles, android concretamente. Actualmente el número de CBIR no es muy grande y no son muy conocidos por los usuarios, por lo que puede verse como una gran oportunidad.
  
  \item Este sistema debe ser capaz de permitir al usuario elegir la imagen consulta de su galería, o mediante la cámara del dispositivo, permitiendo realizar la consulta con dichas imágenes.
  
  \item Los resultados deben ser obtenidos en un periodo de tiempo aceptable. Esto ha de ser así, ya que en caso contrario, la experiencia del usuario se resintiría, lo que podría traducirse en un abandono de la aplicación.  
  
\end{itemize}


\section{Objetivos secundarios}

Una vez desarrollados los principales objetivos del proyecto, se explicarán los secundarios:

\begin{itemize}
  \item Utilizar una base de datos para almacenar los resultados. De esta manera, una vez calculada la primera consulta, el tiempo del resto se reduce drásticamente, lo que se traduce en una mejor experiencia para el usuario.

  \item El tiempo de respuesta ha de ser el mínimo posible, ya que se trabajará con un gran número de imágenes y los calculos asociados a estas.

\end{itemize}
