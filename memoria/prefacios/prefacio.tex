%\chapter*{}
%\thispagestyle{empty}
%\cleardoublepage

%\thispagestyle{empty}

%\input{portada/portada_2}
%\cleardoublepage
%\thispagestyle{empty}

\begin{center}
{\large\bfseries \myTitle}\\
\end{center}
\begin{center}
\myName\\
\end{center}

%\vspace{0.7cm}
\noindent{\textbf{Palabras clave}: \textit{visualización imágenes}, \textit{java 3d}, \textit{3d}, \textit{recuperación imágenes}, \textit{CBIR}}\\

\vspace{0.7cm}
\noindent{\textbf{Resumen}}\\

Debido al avance de la tecnología, cada vez disponemos de más dispositivos con la capacidad de capturar imágenes. Esto ha provocado un importante incremento en las bases de datos de imágenes, lo que a su vez ha propiciado la aparición de metodologías para solventar el problema de la manipulación, gestión y recuperación de dicha información. En este caso cuando hablamos de metodologías nos referimos a los sistemas de recuperación de información, basados fundamentalmente en descriptores de bajo nivel (color, textura, etc.) obtenidos directamente a partir de la imagen. Estos sistemas se denominan CBIR.\\

El objetivo de este proyecto es desarrollar módulos para visualizar la información obtenida mediante el uso de sistemas de recuperación de información, en concreto Java Multimedia Retrieval©. El entorno en el que se representa la información es un entorno tridimensional, en el cual el usuario puede moverse libremente, modificar ciertos parámetros para mejorar su experiencia y obtener datos sobre lo que se muestra en dicho entorno.\\

La principal ventaja de estos módulos respecto a otros es el uso de un entorno en tres dimensiones. La mayoría de los CBIR utilizan visualizaciones en dos dimensiones, lo que puede resultar problemático si se quiere observar el resultado de la consulta usando varios descriptores. Este proyecto cuenta con visualizaciones que solo requieren un único descriptor, aunque pueden usarse con mas de uno, y otras que requieren varios descriptores para funcionar correctamente.\\

Para el desarrollo de los distintos módulos se ha usado Java 3D. Se trata de una API usada para la creación de gráficos 3D mediante el uso del lenguaje de programación Java.\\





%\cleardoublepage


%\thispagestyle{empty}

\newpage


\begin{center}
{\large\bfseries \myEngTitle}\\
\end{center}
\begin{center}
\myName\\
\end{center}

%\vspace{0.7cm}
\noindent{\textbf{Keywords}: \textit{image visualization}, \textit{java 3d}, \textit{3d}, \textit{image retrieval}, 
\textit{CBIR}}\\

\vspace{0.7cm}
\noindent{\textbf{Abstract}}\\

Due to the advancement of technology, more and more devices have the ability to capture images. This has caused a significant increase in image databases, which has led to a significant increase in image databases, which in turn has led to the emergence of methodologies to solve the problem of handling, management and retrieval of such information. In this case when we talk about methodology we refer to the information retrieval systems, based primarily on low-level descriptors (color, texture, etc.) obtained directly from the image.These systems are called CBIR.\\

The aim of this project is to develop modules to display the information obtained by using information retrieval systems, specifically Java Multimedia Retrieval©. The environment in which information is represented is a three-dimensional environment in which the user can move freely through it, modify certain parameters to improve the experience and get data on what is shown in that environment.\\

The main advantage of these modules over others is the use of a three-dimensional environment. Most CBIR use two-dimensional views which can be problematic if you want to see the result of the query using various descriptors. This project has visualizations that only require a single descriptor, but can be used with more than one, and others that require various descriptors to function properly.\\

For the development of the different modules has been used Java 3D. It is an API used for creating 3D graphics using the Java programming language.\\

\newpage

%AUTORIZACION

%\chapter*{}
\thispagestyle{empty}

\noindent\rule[-1ex]{\textwidth}{2pt}\\[4.5ex]

Yo, \textbf{\myName}, alumno de la titulación \myDegree{} de la \textbf{\myFaculty} con DNI 75928287C, autorizo la
ubicación de la siguiente copia de mi Trabajo Fin de Grado en la biblioteca del centro para que pueda ser
consultada por las personas que lo deseen.

\vspace{6cm}

\noindent Fdo: \myName

\vspace{2cm}

\begin{flushright}
\myLocation a \myTime.
\end{flushright}

\newpage

%\chapter*{}
\thispagestyle{empty}

\noindent\rule[-1ex]{\textwidth}{2pt}\\[4.5ex]

\textbf{\myProf}, profesor del \myDepartment de la \myUni.

\vspace{0.5cm}

\textbf{Informa}

\vspace{0.5cm}

Que el presente trabajo, titulado \textit{\textbf{\myTitle}},
ha sido realizado bajo su supervisión por \textbf{\myName}, y autorizamos la defensa de dicho trabajo ante el tribunal
que corresponda.

\vspace{0.5cm}

Y para que conste, expide y firma el presente informe en \myLocation a \myTime.

\vspace{1cm}

\textbf{El tutor:}

\vspace{5cm}

\noindent \textbf{\myProf}

\chapter*{Agradecimientos}
\thispagestyle{empty}

       \vspace{1cm}


Quiero dar las gracias a mi familia, amigos y profesores porque sin ellos nada de esto hubiera sido posible.

